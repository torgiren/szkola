\documentclass[12pt,a4paper]{article}
\usepackage[utf8]{inputenc}
\usepackage{amsmath}
\usepackage{listings}
\usepackage{verbatim}
\usepackage{graphicx} 
\oddsidemargin 0cm
\marginparwidth 0cm
\hoffset 0cm
\usepackage{polski}
\begin{document} 
\large
\begin{tabular}{|c|c|c|c|}
\hline
\multicolumn{4}{|l|}{Temat:}\\
\multicolumn{4}{|c|}{Wyznaczanie wartości i~wektorów własnych macierzy symetrycznej}\\
\multicolumn{4}{|c|}{metodą potęgową z~redukcją Hotellinga}\\
\hline
\multicolumn{1}{|l}{Wykonał:}&\multicolumn{1}{|l}{Wydział:}&\multicolumn{1}{|c}{Kierunek}&\multicolumn{1}{|l|}{Grupa:}\\
Marcin Fabrykowski&FiIS&Inf. Stos.&grupa 3\\
\hline
\end{tabular}
\normalsize
\vspace{2cm}
\begin{enumerate}
\item Wstęp\\
Metoda potęgową z~wykorzystaniem redukcji Hotellinga jest skuteczna tylko dla macierzy symetrycznych.
Aby wyznaczyć wartości własne należy wykorzystać następujący algorytm:
\begin{enumerate}
\item Ustalamy numer poszukiwanej wartości własnej $k=1,2,\dots,n$
\item Ustalamy wektor startowy $x_0=[1,2,\dots,n]$
\item Następnie iteracyjnie wykonujemy kolejne przybliżenia $\lambda$:
\begin{enumerate}
\item $x_{i+1}=W_kx_i$
\item $\lambda_i=\dfrac{x_{i+1}^T}{x_i^Tx_i}$
\item $x_{i+1}=\dfrac{x_{i+1}}{\|x_{i+1}\|_2}$
\item $x_i=x_{i+1}$
\end{enumerate}
po wykonaniu powyższego, wykonujemy redukcję macierzy W: $W_{k+1}=W_k-\lambda_kx_kx_k^T$
\end{enumerate}
Wartość $\lambda_k$ reprezentuje $k$-tą wartość własną macierzy.
\item Wykonanie\\
Wyznaczamy wektory własne macierzy A, gdzie $A_{ij}=A_{ji}=\sqrt{i+j}$ przy użycia biblioteki Numerical Reciples, a~następnie porównujemy je z~wartościami otrzymanymi metodą Hotellinga.\\
Zadanie to realizuje poniższy program:
\lstinputlisting[language=C++,caption=main.cpp,breakatwhitespace=true,basicstyle=\footnotesize,breaklines=true]{main.cpp}
Czego wynikiem jest:\\
\footnotesize
\verbatiminput{date.txt}
\normalsize
\item Wnioski\\
Metoda potęgowa jest nieco wolniejsza niż ta z~Numerical Reciples, jednakże ta druga nie radzi sobie z~większymi macierzami. Przestaje działać przy $n>=180$. Natomiast metoda potęgowa daje sobie radę nawet z~rozmiarami $n=500$, co czyni ją bardzo przydatną przy większych macierzach.
\end{enumerate}
\end{document}
