\hypertarget{main_8cpp}{
\section{main.cpp File Reference}
\label{main_8cpp}\index{main.cpp@{main.cpp}}
}
a to jest proba 

{\tt \#include $<$iostream$>$}\par
\subsection*{Functions}
\begin{CompactItemize}
\item 
int \hyperlink{main_8cpp_a5838489380856ef49519c2380de3e7d}{dodaj} (int a, int b)
\item 
\hypertarget{main_8cpp_e66f6b31b5ad750f1fe042a706a4e3d4}{
int \hyperlink{main_8cpp_e66f6b31b5ad750f1fe042a706a4e3d4}{main} ()}
\label{main_8cpp_e66f6b31b5ad750f1fe042a706a4e3d4}

\begin{CompactList}\small\item\em funkcja main \item\end{CompactList}\end{CompactItemize}


\subsection{Detailed Description}
a to jest proba 

\begin{Desc}
\item[Author:]Fabrykowski \end{Desc}


\subsection{Function Documentation}
\hypertarget{main_8cpp_a5838489380856ef49519c2380de3e7d}{
\index{main.cpp@{main.cpp}!dodaj@{dodaj}}
\index{dodaj@{dodaj}!main.cpp@{main.cpp}}
\subsubsection[dodaj]{\setlength{\rightskip}{0pt plus 5cm}int dodaj (int {\em a}, \/  int {\em b})}}
\label{main_8cpp_a5838489380856ef49519c2380de3e7d}


funkcja dodaje 2 liczny. \begin{Desc}
\item[Parameters:]
\begin{description}
\item[{\em a}]pierwsza liczba \item[{\em b}]druga liczba \end{description}
\end{Desc}
\begin{Desc}
\item[Returns:]suma liczb \end{Desc}
\begin{Desc}
\item[\hyperlink{todo__todo000001}{Todo}]do zrobienia \end{Desc}
