\documentclass[a4paper,12pt]{article}
\usepackage[utf8]{inputenc}
\usepackage{polski}
\author{Marcin TORGiren Fabrykowski}
\title{Bazy danych II\\ projekt A-2}
\begin{document}
\maketitle
\newpage
\tableofcontents
\newpage
\section{Tematyka projektu}
Celem programu jest konstrukcja zapytań SQL, bazując na dynamicznej liście baz, tabel i kolumn.\\
Program pozwala użytkownikowi nieznającemu języka zapytań SQL na budowanie zapytań do bazy poprzez przyjazny interfejs tekstowy.
\section{Wykorzystane technologie}\
\begin{description}
\item[baza danych]\hfill \\
	MS SQL Server
\item[język programowania]\hfill \\
	Python
\item[interface bazy danych]\hfill \\
	pymssql	
\item[interface użytkownika]\hfill \\
	curses
\end{description}
\section{Ogólny zarys}
Po uruchomieniu, program łączy się z bazą danych używając moduły pymssql.\\
Dane do logowania zostały zapisane na sztywno w programie.\\
Łączenie odbywa się przez tunel ssh prowadzący z localhost:1433 -> hostvirtualny:1433.\\
Po udanym połączeniu następuje pobranie listy baz danych, oraz wyświetlenie jej w postaci przyjaznego menu.\\
Po wybraniu bazy danych, program wysyła zapytanie ustawiające wybraną bazę jako używaną bazę danych.\\
Następnie pobierana jest lista tabel w aktualnej bazie danych i zostaje ona wyświetlona jako menu.\\
Dla wybranej tabeli, zostaje pobrana lista kolumn ... do dopisania


\end{document}
