\documentclass[a4paper,12pt]{article}
\usepackage[utf8]{inputenc}
\usepackage{polski}
\author{Marcin TORGiren Fabrykowski}
\title{Bazy danych II\\ projekt A-2\\Dynamiczna konstrukcja warunków wyszukiwania.}
\begin{document}
\maketitle
\newpage
\tableofcontents
\newpage
\section{Tematyka projektu}
Celem programu jest konstrukcja zapytań SQL, bazując na dynamicznej liście baz, tabel i kolumn.\\
Program pozwala użytkownikowi nieznającemu języka zapytań SQL na budowanie zapytań do bazy poprzez przyjazny interfejs tekstowy.
\section{Wykorzystane technologie}\
\begin{description}
\item[baza danych]\hfill \\
	MS SQL Server
\item[język programowania]\hfill \\
	Python
\item[interface bazy danych]\hfill \\
	pymssql	
\item[interface użytkownika]\hfill \\
	curses
\end{description}
\section{Komunikacja z~bazą}
\subsection{Łączenie do bazy}
Program łączy się z bazą danych używając moduły pymssql.\\
\subsection{Wybór bazy danych}
Pobranie listy baz danych odbywa się przy pomocy funkcji \textit{sp\_databases}.\\
\subsection{Listowanie tabel}
Lista tabel w aktualnej bazie danych pobierana jest poprzez wykonanie zapytania do tabeli \textit{INFORMATION\_SCHEMA} pobierając kolumnę \textit{TABLES} filtrując typ tabel jako \textit{BASE\_TABLE}.
\subsection{Listowanie kolumn}
Dla wybranej tabeli, zostaje pobrana lista kolumn poprzez zapytanie do tabeli \textit{INFORMATION\_SCHEMA}, do kolumny \textit{COLUMNS}, filtrując po nazwie tabeli \textit{TABLE\_NAME}
\section{Schemat działania}
\subsection{Uruchomienie programu}
Program uruchamiany jest z~dwoma parametrami, którymi są odpowiednio: adres serwera i~port MS SQL Servera.
\subsection{Przygotowanie środowiska}
Po uruchomieniu, program łączy się z~bazą danych, a~następnie przygotowywany jest przyjazny interfejs użytkownika.\\
Pobierana jest lista baz danych z~serwera
\subsection{Wybór bazy danych}
Lista baz danych wyświetlana jest użytkownikowi w~postaci menu.\\
Po wybraniu bazy danych, pobierana zostaje lista tabel w~wybranej bazie.\\
\subsection{Wybór tabeli}
Pobrana lista tabel zostaje przedstawiona użytkownikowi w~menu.
\subsection{Wybór kolumn}
Następnie w~pętli następuje pobieranie listy kolumn w~tabeli i~wyświetlenie ich w~menu, pozwalając użytkownikowi na wielokrotne wybieranie kolumn.
Wybrane kolumny są dodawane do listy kolumn do wyselekcjonowania.\\
Zakończenie wybierania kolumn następuje po naciśnięciu klawisza F10.\\
\subsection{Wybór kryteriów}
Kolejnym etapem jest wybieranie kryteriów wyboru. Użytkownikowi przedstawiana jest lista kolumn, z~których użytkownik wybiera kolumnę. Następnie z~nowego menu, użytkownik wybiera sposób filtrowania.
Następnie użytkownik w~nowym oknie wpisuje wartość używaną do filtrowania.\\
W~przypadku wyboru kolejnego pola do filtracji, użytkownik zostaje poproszony dodatkowo, o~sprecyzowanie logicznego warunku łączenia kryteriów.
Wybór kryteriów kończy się po naciśnięciu klawisza F10.\\
\subsection{Generowanie wyniku}
Po zakończeniu wprowadzania danych, program generuje zapytanie do bazy danych na podstawie podanych wartości, oraz je wykonuje i~wyświetla wyniki zapytania oraz samo zapytanie.
\end{document}
