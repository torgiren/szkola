\documentclass[10pt]{beamer}
\usepackage[utf8]{inputenc}
\usepackage{polski}
\usepackage{verbatim}
\author{Marcin TORGiren Fabrykowski}
\title{Streaming video}
\usetheme{Warsaw}
\begin{document}
\begin{frame}
	\titlepage
\end{frame}
\begin{frame}
	\frametitle{O czym będziemy mówić...}
	\tableofcontents
\end{frame}
\section{Trochę teorii}
\subsection{Co to jest streaming}
\begin{frame}
	\frametitle{Co to jest streaming}
	\begin{block}<1->
	{streaming}
	przesyłanie ciągłych danych multimedialnych, najczęściej, przez internet
	\end{block}
	\begin{block}<2->
	{rodzaje strimingów}
	\begin{itemize}
		\item<3-> live (np: telewizja w akademiku)
		\item<4-> VOD (np: wypożyczalnia internetowa)
	\end{itemize}
	\end{block}
	\begin{block}<5->
	{protokoły transmisji}
	\begin{itemize}
		\item<6-> UDP
		\item<7-> HTTP
		\item<8-> RTSP
	\end{itemize}
	\end{block}
\end{frame}
\subsection{Kodeki}
\begin{frame}
	\frametitle{Kodeki}
	Testowałem na przykładowym odcinku Świata wg. Bundych.
	\begin{block}<1->
	{z użyciem kodeka}
	\uncover<2->
	{
		film zajmuje 178MB
	}
	\end{block}
	\begin{block}<3->
	{bez kodeka}
	\uncover<4->
	{
		film zajmuje 12GB
	}
	\end{block}
\end{frame}
\begin{frame}
	\frametitle{Przykładowe kodeki}
	\begin{block}<1->
	{najpopularniejsze kodeki video}
	\begin{itemize}
	\item MPEG-1/2/4
	\item H264
	\item FLV1
	\end{itemize}
	\end{block}
	\begin{block}<2->
	{najpopularniejsze kodeki audio}
	\begin{itemize}
	\item MPEG auidio
	\item MPEG-3
	\item FLAC
	\item A52
	\end{itemize}
	\end{block}
\end{frame}
\subsection{Kontenery}
\begin{frame}
	\frametitle{Kontenery}
	\begin{block}<1->
	{co to jest}
	służą do pakowania kilku strumieni w jednym opakowaniu\\
	działają jak multiplexer
	\end{block}
	\begin{block}<2->
	{najpopularniejsze kontenery}
	\begin{itemize}
	\item AVI
	\item Matroska
	\item MPEG
	\item MP4
	\end{itemize}
	\end{block}
\end{frame}
\section{Praktyka}
\subsection{Live stream}
\begin{frame}[fragile]
	\frametitle{Live stream}
	\begin{block}<1->
	{stream z kamery/karty TV}
	\footnotesize
	\begin{verbatim}
	cvlc -v --ttl 5 --rt-priority v4l2:///dev/video1:width=640:height=480
	:input-slave=alsa://hw:0,2 :samplerate=44100
	:sout=#transcode{vcodec=h264,vb=512,fps=25,scale=1,acodec=mp3,ab=128,
	channels=1,audio-sync,deinterlace,threads=2}:duplicate{dst=std
	{access=http,mux=ts,dst=0.0.0.0:3344}}"
	\end{verbatim}
	\end{block}
	\begin{block}<2->
	{odtwarzanie}
	\begin{itemize}
		\item mplayer http://server:3344
		\item vlc http://server:3344
	\end{itemize}
	\end{block}
\end{frame}
\subsection{VOD}
\begin{frame}[fragile]
	\frametitle{VOD}
	\begin{block}<1->
	{uruchomienie}
	\footnotesize
	\begin{verbatim}
	vlc  -v --color -I telnet --telnet-password videolan --telnet-port 4121
	--rtsp-host 0.0.0.0 --rtsp-port 3344
	\end{verbatim}
	\end{block}
	\begin{block}<2->
	{telnotowanie}
	\begin{verbatim}
	telnet localhost 4212
	\end{verbatim}
	\end{block}
	\begin{block}<3->
	{konfiguracja}
	\begin{verbatim}
	new film1 vod enabled
	setup film1 input film.mpg
	\end{verbatim}
	\end{block}
	\begin{block}<4->
	{odtwarzanie}
	\begin{itemize}
		\item vlc rtsp://server:3344/film1
		\item mplayer rtsp://server:3344/film1
	\end{itemize}
	\end{block}
\end{frame}
\end{document}
