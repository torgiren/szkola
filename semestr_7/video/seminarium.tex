\documentclass[10pt]{beamer}
\usepackage[utf8]{inputenc}
\usepackage{polski}
\author{Marcin TORGiren Fabrykowski}
\title{Streaming video}
\usetheme{Warsaw}
\begin{document}
\begin{frame}
	\titlepage
\end{frame}
\begin{frame}
	\frametitle{O czym będziemy mówić...}
	\tableofcontents
\end{frame}
\section{Trochę teorii}
\subsection{Co to jest streaming}
\begin{frame}
	\frametitle{Co to jest streaming}
	\begin{block}<1->
	{streaming}
	przesyłanie ciągłych danych multimedialnych, najczęściej, przez internet
	\end{block}
	\begin{block}<2->
	{rodzaje strimingów}
	\begin{itemize}
		\item<3-> live (np: telewizja w akademiku)
		\item<4-> VOD (np: wypożyczalnia internetowa)
	\end{itemize}
	\end{block}
	\begin{block}<5->
	{protokoły transmisji}
	\begin{itemize}
		\item<6-> UDP
		\item<7-> HTTP
		\item<8-> RTSP
	\end{itemize}
	\end{block}
\end{frame}
\subsection{Kodeki}
\begin{frame}
	\frametitle{Kodeki}
	Testowałem na przykładowym odcinku Świata wg. Bundych.
	\begin{block}<1->
	{z użyciem kodeka}
	\uncover<2->
	{
		film zajmuje 178MB
	}
	\end{block}
	\begin{block}<3->
	{bez kodeka}
	\uncover<4->
	{
		film zajmuje 12GB
	}
	\end{block}
\end{frame}
\begin{frame}
	\frametitle{Przykładowe kodeki}
	\begin{block}<1->
	{najpopularniejsze kodeki video}
	\begin{itemize}
	\item MPEG-1/2/4
	\item H264
	\item FLV1
	\end{itemize}
	\end{block}
	\begin{block}<2->
	{najpopularniejsze kodeki audio}
	\begin{itemize}
	\item MPEG auidio
	\item MPEG-3
	\item FLAC
	\item A52
	\end{itemize}
	\end{block}
\end{frame}
\subsection{Kontenery}
\begin{frame}
	\frametitle{Kontenery}
\end{frame}
\section{Streaming unicastowy}
\section{VOD}
\end{document}
