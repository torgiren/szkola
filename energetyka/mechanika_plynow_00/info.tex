\documentclass[a4paper,12pt]{article}
\usepackage{polski}
\usepackage[utf8]{inputenc}
\usepackage{commath}
\title{Mechanika płynów - lab 0}
\author{Marcin TORGiren Fabrykowski}
\begin{document}
\maketitle
\newpage
\tableofcontents
\newpage
\section{Informacje organizacyjne}
\subsection{Prowadzacy}
K. Filek, p 107, A1
\subsection{Skrypt}
Mechanika płynów z elementami pomiaroznawstwa, K. Filek,  Rodczymialski(czy jakoś tak), Wacławik, 1990r
\subsection{Grupa}
grupa Id
Osoby: Kamil Płonka, Kamil Górczyński, Jakub Siewierski
plonka-x@o2.pl
\section{Organizacja zajęć}
\subsection{Ćwiczenia}
\subsubsection{Nr 2. Pomiar natężenia przepływu (ilość/czas) (to samo co wydatek,strumień)}
	$Q=\dfrac{V}{t}$\\
	Kryza pomiarowa, zwężka Venturiego, sonda prandtla\\
	$Q_{k,z}=\dfrac{\pi d^2_{k,z}}{4}\alpha_{k,z}\sqrt{\dfrac{2\Delta p_{k,z}}{\rho}}$\\
	$\rho_{pow}= 1.15\dfrac{kg}{m^3}$\\
	$p_d=\dfrac{\rho V^2}{2}$\\
	$V=\sqrt{\dfrac{2p_d}{\rho}}$\\
	$Q=V_{sr}\dfrac{\pi d^2}{4}$\\
	$V_{sr}=0.8V_{max}$\\
\subsubsection{Nr 8. Pomiar współczynnika oporu liniowego}
	$\Delta p=\lambda \dfrac{L}{D}\dfrac{\rho V^2}{2}$\\
	$\lambda=\dfrac{2D \Delta p}{L\rho V^2}$\\
	Zmiana średnicy: $V=0.8 \sqrt{2p_d}{\rho} \dfrac{D_p}{D}$\\
	$Re=\dfrac{V_{sr}D}{\nu}$\\
	$\nu=1.6*10^{-5} \dfrac{m^2}{s}$\\
\subsubsection{Nr 16. Pomiar współczynnika przepływu w ośrodku porowatym}
	Prawo Darcy'ego: $\vec{V}=-\dfrac{k}{\mu} \mathrm{grad}p$\\
	$k=\dfrac{\mu Q \ln{\dfrac{r_z}{r_w}}}{2\pi L \Delta p}$\\
\subsubsection{Pomiar lepkości powietrza}
	$Q=\dfrac{\pi \Delta p D^4}{128 \mu L}$\\
	$\mu=\dfrac{\pi \Delta p D^4 t}{128 VL}$\\
	$Re=\dfrac{V_{sr}D}{\nu}$\\
	$Re_{graniczne}=2300$\\
	mniej - laminarny,\\
	więcej - nie wiemy, ale raczej jest nielaminarny\\
\subsection{Harmonogram}
\begin{tabular}{c|c|c|c|c}
&I&II&III&IV\\
A&2&3&4&8\\
D&8&2&3&4\\
\end{tabular}
\end{document}
