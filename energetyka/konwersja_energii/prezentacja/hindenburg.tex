\documentclass{beamer}
\usepackage{polski}
\usepackage[utf8]{inputenc}
\usepackage{graphicx}
\usetheme{Warsaw}
\usepackage[normalem]{ulem}
\author{Marcin Fabrykowski}
\title{Przyczyny katastrofy Hindenburga}
\begin{document}
\begin{frame}
\maketitle
\end{frame}
\section{Podstawowe informacje}
\subsection{Czym był Hindenburg?}
\begin{frame}
	\frametitle{Czym był Hindenburg}
	\begin{block}{Pełna nazwa}<1->
		LZ 129 Hindenburg
	\end{block}
	\begin{block}{Zastosowanie}<2->
		\begin{itemize}
			\item cywilne
			\item transport transatlantycki
		\end{itemize}
	\end{block}
\end{frame}
\subsection{Czym była seria Zeppelins?}
\begin{frame}
	\frametitle{Co znacza LZ?}
	\begin{block}{LZ}
		Luftschiff Zeppelin
	\end{block}
		\begin{block}{Zeppelin}<2->
		\begin{itemize}
			\item<3-> seria sterowców
			\item<4-> Ferdinand von Zeppelin
		\end{itemize}
	\end{block}
\end{frame}
\subsection{Czym był sterowiec?}
\begin{frame}
	\frametitle{Czym był sterowiec?}
	\includegraphics[width=0.8\paperwidth]{Hindenburg}
\end{frame}
\subsection{LZ 129 Hindenburg}
\begin{frame}
	\frametitle{LZ 129 Hindenburg}
	\begin{block}{wymiary}
	Długość: 245m\\
	Średnica 41m
	\end{block}
	\uncover<2->{
		\begin{center}
		\includegraphics[width=0.5\paperwidth]{Hind_size}
		\end{center}
	}
\end{frame}
\begin{frame}
	\frametitle{LZ 129 Hindenburg}
	\begin{block}{Pojemność}
		\begin{itemize}
			\item Pasażerowie: 50 - 72
			\item Załoga: 61
		\end{itemize}
	\end{block}
	\begin{block}{Napęd}
		4x silnik diesla 1200KM
	\end{block}
	\begin{block}{Prędkość}
		Prędkość maksymalna: 135km/h
	\end{block}
	\begin{block}{Służba}
		\begin{itemize}
		\item Wszystkich lotów: 63
		\item Transatlantyckich: 17
		\end{itemize}
	\end{block}
\end{frame}
\section{Katastrofa Hindenburgs}
\subsection{Przebieg}
\begin{frame}
	\frametitle{Katastrofa}
	\begin{block}{Przebieg}
	(tutaj pokazać film)
	\end{block}
\end{frame}
\subsection{Czas i~miejsce}
\begin{frame}
	\frametitle{Informacje}
	\begin{block}{Kiedy?}
	6 Maja 1937
	\end{block}
	\begin{block}{Gdzie?}
	Lakehurst, New Jersey, USA
	\end{block}
\end{frame}
\subsection{Statystyki}
\begin{frame}
	\frametitle{Statystyki}
	\begin{block}{Osoby na pokładzie}
		\begin{itemize}
			\item Pasażerowie: 36
			\item Załoga: 61
		\end{itemize}
	\end{block}
	\begin{block}{Ofiary}
		\begin{itemize}
			\item Pasażerowie: 13
			\item Załoga: 22
			\item Obsługa naziemna: 1
		\end{itemize}
	\end{block}
\end{frame}
\section{Możliwe przyczyny}
\begin{frame}
	\frametitle{Możliwe przyczyny katastrofy}
	\begin{itemize}
		\item \alt<2->{\sout{sabotaż}}{sabotaż}
		\item \alt<3->{\sout{zmęczenie konstrukcji}}{zmęczenie konstrukcji}
		\item \alt<4->{\sout{awaria silnika}}{awaria silnika}
		\item \alt<5->{\sout{łatwopalne poszycie}}{łatwopalne poszycie}
		\item \alt<6->{\sout{nieszczelność zbiorników}}{nieszczelność zbiorników}
		\item \alt<7->{\sout{nagromadzenie ładunku elektrycznego}}{nagromadzenie ładunku elektrycznego}
		\item zerwanie liny wzmacniającej
	\end{itemize}
\end{frame}
\begin{frame}
	\frametitle{Dziękuje za uwagę}
	\begin{beamercolorbox}[center,shadow=true,rounded=true,]{note} 
        \Huge{Pytania?}
	\end{beamercolorbox} 
\end{frame}
\end{document}
