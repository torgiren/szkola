\documentclass[a4paper,12pt]{article}
\usepackage{polski}
\usepackage[utf8]{inputenc}
\author{Marcin Fabrykowski}
\title{Ledicator - oświetlenie komputera wykonane w technologii led}
\begin{document}
\maketitle
\newpage
\tableofcontents
\newpage
\section{Opis produktu}
	Ledicator jest oświetleniem komputera wykonanym w technologii led, z możliwością sterowania poziomem wskaźnika.
	Oświetlenie wykonane jest z szeregu 20 diod led.
	Zamontowane zostały diody w trzech kolorach - zielone, żółte oraz czerwone.\\
	Urządzenie podłączane jest do komputera za pomocą portu USB.
\section{Dane techniczne}
	\begin{itemize}
	\item Napięcie zasilania: 5V DC
	\item Złącze: USB
	\end{itemize}
\section{Sterowanie}
	\subsection{Wysokopoziomowe}
		Sterowanie odbywa się przy użyciu programu sterującego załączonego do urządzenia na płycie CD bądź pobranego ze strony internetowej.
		Program pozwala na wybranie jednego z dostępnych schematów oświetlenia i animacji, bądź prezentowania monitorowanych wartości, takich jak obciążenie procesora, zużycie RAM bądź zajętość dysku.
	\subsection{Niskopoziomowe}
		Komunikacją z urządzeniem zajmuje się moduł jądra umieszczony na płycie CD bądź pobrany ze strony internetowej.	
		Sterowanie urządzeniem odbywa się poprzez przesyłanie wartości z przedziału od 0 do 1048576 do deskryptorów modułu obsługującego.
		Przesłana wartość jest interpretowana jako mapa bitowa, gdzie kolejne bity odpowiadają za tryb zapalenie kolejnych diod.
\section{Schemat}
	Tutaj powinny być schematy układu sterującego.
\section{Instrukcja obsługi}
	
\end{document}
