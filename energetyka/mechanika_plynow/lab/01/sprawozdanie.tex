\documentclass[a4paper,12pt]{article}
\usepackage[utf8]{inputenc}
\usepackage{polski}
\usepackage{amsmath}
\title{Mechanika płynów\\
Laboratorium 01\\
Badanie lepkości powietrza}
\author{Marcin Fabrykowski\\Kamil Płonka\\Kamil Górczyński\\Jakub Siewierski}
\begin{document}
\maketitle
\newpage
\section{Wstęp teoretyczny}
Lepkość jest to opór wewnętrzny płynu, wynikający z powstawania wirów wewnątrz płynu na skutek przekazywania pędu pomiędzy warstwami o różnych prędkościach przepływu.\\
Aby wyznaczyć teoretyczny współczynnik lepkości, stosujemy wzór na natężenie przepływu:
$$Q=S*v$$
Będziemy badać przepływ przez rurociąg o przekroju kołowym, dlatego pole przekroju przyjmujemy:
$$S=\dfrac{\pi D^2}{4}$$
Natomiast prędkość maksymalna cieczy w rurociągu wyznaczamy ze wzoru:
$$v=\dfrac{\Delta pD^2}{4*8*\mu*l}$$
gdzie:\\
$ Q$ - Natężenie przepływu\\
$ S$ - przekrój rurociągu\\
$ v$ - prędkość cieczy\\
$\Delta p $ - spadek ciśnienia\\
$ D$ - średnica rurociągu\\
$ \mu$ - lepkość płynu\\
$ l$ - długość rurociągu\\
Podstawiając powyższe równania do wzoru, otrzymujemy:
$$Q=\dfrac{\pi\Delta pD^4}{128\mu l}$$
Wyznaczając z powyższego interesującą nas lepkość, otrzymamy:
$$\mu=\dfrac{\pi\Delta pD^4}{128Ql}$$
Zauważając, że natężenie przepływu jest równe:
$$Q=\dfrac{V}{t}$$
Możemy wykorzystać tą zależność i wstawić wartości, które jesteśmy w stanie zmierzyć prostymi przyrządami. Ostateczny wzór na wyznaczenie lepkości płynu wygląda następująco:
$$\mu=\dfrac{\pi\Delta pD^4t}{128Vl}$$

Aby powyższy był prawdziwy, przepływ w rurociągu musi być laminarny.\\
Aby sprawdzić charakter przepływu, należy wyznaczyć \textit{liczbę Reynoldsa}.\\
Liczba ta wyraża się wzorem:
$$Re=\dfrac{v_{sr}D}{\mu}$$
gdzie:\\
$Re$ - liczba Reynoldsa\\
$v_{str}$ - prędkość średnia cieczy, wyrażona stosunkiem $0.8v_{max}$\\

Możemy przyjąć, że dla wartości $Re<2300$ przepływ miał charakter laminarny, natomiast w przeciwnym wypadku charakter burzliwy.
\section{Wykonanie ćwiczenia}
Ćwiczenie wykonujemy przepuszczając przez rurociąg pewną objętość płynu, mierząc jednocześnie czas w jakim to nastąpiło oraz spadek ciśnienia w rurociągu na badanym odcinku.\\
Następnie, wykorzystując wyprowadzone wcześniej wzory, dokonujemy obliczeń lepkości oraz liczby Reynoldsa.\\
Poniższa tabela przedstawia zmierzone wartości oraz otrzymane wyniki:\\
\begin{tabular}{|c|c|c|c|c|c|c|}
\hline
L.p&t[s]&$\Delta p$[Pa]&V[$mm^3$]&$\mu$[Pa*s]&$v_{sr}$[m/s]&Re\\
\hline
1&$8.1$&$385$&$390$&$18.1*10^{-6}$&$6,38$&$1092,19$\\
2&$7.93$&$373$&$361$&$18.5*10^{-6}$&$6.03$&$1007.76$\\
3&$8.01$&$345$&$343$&$18.2*10^{-6}$&$5.67$&$964,06$\\
4&$8.18$&$349$&$348$&$18.5*10^{-6}$&$5.63$&$940.65$\\
5&$8.03$&$334$&$330$&$18.4*10^{-6}$&$5.21$&$917.18$\\
6&$8.26$&$321$&$325$&$18.4*10^{-6}$&$5.21$&$874.79$\\
7&$7.99$&$309$&$306$&$18.2*10^{-6}$&$5.07$&$860.98$\\
8&$8.17$&$297$&$301$&$18.2*10^{-6}$&$4.88$&$828.96$\\
9&$7.93$&$283$&$280$&$18.1*10^{-6}$&$4.69$&$801.09$\\
10&$8.16$&$272$&$275$&$18.2*10^{-6}$&$4.46$&$757.39$\\
\hline
&&&średnia:&$18.3*10^{-6}$&&\\
\hline
\end{tabular}\\
Przy danych stanowiska pomiarowego:\\
D=$3.1$mm\\
L=$1$m\\
\section{Wnioski}
Otrzymany wynik średni, wynoszący $18.3*10^-6$Pa*s, jest większy niż tablicowy, który wynosi $17.08*10^-6$Pa*s.\\
Przypuszczam, że różnica ta wynika z niedokładności pomiarowych spowodowanych niedoskonałym czynnikiem ludzkim odmierzającym czas i/lub mierzącym objętości wody.\\
Wynik różniący się o 7\% możemy uznać za zadowalający

Jako, iż dla każdego pomiaru, liczba Reynoldsa jest mniejsza niż graniczna, tj. 2300, stwierdzamy, że przepływ w każdym przypadku był laminarny, co daje nam podstawy uważać otrzymaną wartość lepkości za prawdziwą.
\end{document}
