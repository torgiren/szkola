\documentclass[a4paper, 12pt]{article}
\usepackage{polski}
\usepackage[utf8]{inputenc}
\usepackage{graphicx}
\usepackage[T1]{fontenc}
\author{Marcin Fabrykowski}
\title{Geometria komputerowa\\Laboratorium 01}
\begin{document}
\maketitle
\newpage
Złoty podział jest podziałem szeroko wykorzystywany w wielu dziedzinach życia, od malarstwa, przed format książek, po laboratoria z geometrii komputerowej\\
Powstrzymam się tutaj od wklejania dużej ilości teorii z Wikipedii.\\


\section{Podział odcinka}
Jednym ze sposobów na wyznaczenie złotego podziału, jest podział odcinka.\\
\begin{figure}
\includegraphics{odcinek}
\caption{Podział odcinka}
\label{fig:odcinek}
\end{figure}
Sposób postępowania:
\begin{enumerate}
\item Tworzymy kwadrat o boku $a = |AC|$
\item Wyznaczamy pkt $B$ będący połową odcinka $AC$
\item Kreślimy odcinek $BE$
\item Odkładamy odcinek $BE$ od punku $B$ na prostej przechodzącej przez punkty $A$ i $C$
\item Stosunek długości $|AC|$ do $|CD|$ jest równy złotej liczbie $\frac{\sqrt{5} + 1}{2}$
\end{enumerate}
Reprezentacja graficzna została przedstawiona na rys. \ref{fig:odcinek}.

\section{Konstrukcja Odoma}
Drugim sposobem na wyznaczenie złotego podziału jest konstrukcja Odoma.
Sposób postępowania:
\begin{enumerate}
\item Rysujemy trójkąt równoboczny
\item Opisujemy na nim okrąg
\item Wyznaczały punkty środkowe odcinków $AC$ oraz $BC$
\item Wyznaczamy punkt $F$ będący przecięciem prostej przechodzącej przed punkty $D$ oraz $E$ oraz opisanego okręgu
\item Stosunek $DE$ do $EF$ jest złotym podziałem
\end{enumerate}
Reprezentacja graficzna została przedstawiona na rys. \ref{fig:wpisany}.

\begin{figure}
\includegraphics{wpisany}
\caption{Konstrukcja Odoma}
\label{fig:wpisany}
\end{figure}
\end{document}
