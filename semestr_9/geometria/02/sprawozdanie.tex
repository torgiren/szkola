\documentclass[a4paper, 12pt]{article}
\usepackage{polski}
\usepackage[utf8]{inputenc}
\usepackage{graphicx}
\usepackage[T1]{fontenc}
\author{Marcin Fabrykowski}
\title{Geometria komputerowa\\Laboratorium 02}
\begin{document}
\maketitle
\newpage
Powtarzając za stroną p. prof. Stanisława Kaprzyka: "W geometrii abstrakcyjnej pojęcie incydencji jest relacją między punktami i liniami prostymi, która ma specjalne znaczenie."\\
Poniżej przedstawię graficzne dowody dla twierdzeń Cevy oraz Eulera, których treści można znaleźć na stronie w/w profesora

\begin{figure}
\includegraphics{cevy}
\caption{Twierdzenie Cevy}
\end{figure}
\begin{figure}
\includegraphics{euler}
\caption{Prosta Eulera}
\end{figure}
\end{document}
