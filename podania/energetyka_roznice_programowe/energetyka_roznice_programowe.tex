\documentclass[a4paper,12pt]{article}
\usepackage[utf8]{inputenc}
\usepackage{polski}
\usepackage{multicol}
\usepackage{fullpage}
\marginparwidth = 25pt
\begin{document}
\begin{center}
\Large{Różnice programowe względem syllabusa 2012/2013}
\end{center}
\vspace{20pt}
\textbf{Różnice programowe}\\
Semestr I:
\begin{enumerate}
\item Podstawy energetyki (różnica)
\end{enumerate}
Semestr II:
\begin{enumerate}
\setcounter{enumi}{-1}
\item Brak
\end{enumerate}
Semestr III:
\begin{enumerate}
\item Budownictwo i fizyka cieplna budowli (różnica)
\item Mechanika płynów (tylko lab.)
\end{enumerate}
Semestr IV:
\begin{enumerate}
\item Inżynieria materiałowa w energetyce (różnica)
\item Komputerowe systemy operacyjne (różnica)
\item Transport ciepła i masy (różnica)
\item Konwersja energii (tylko egz.)
\end{enumerate}
Semestr V:
\begin{enumerate}
\item Dozymetria i ochrona radiologiczna z detekcją promieniowania (różnica)
\item Wybrane zagadnienia polityki energetycznej, prawa i normalizacji (tylko egz.)
\item Podstawy projektowania (różnica)
\item Chemia fizyczna (różnica)
\end{enumerate}
\vspace{20pt}
\textbf{Zaliczone przedmioty, nie występujące w nowym programie studiów:}
\begin{enumerate}
\item Metody oceny efektywności ekonomicznej
\item Systemy energetyczne
\item Konwersja paliw i energii I
\item Podstawy energetyki jądrowej
\item Technologie energetyczne
\end{enumerate}
\newpage
\textbf{Zaliczone przedmioty:}
\begin{multicols}{2}
\begin{enumerate}
\item Chemia I
\item Matematyka I
\item Fizyka I
\item Informatyka
\item Geometria wykreślna i grafika inżynierska
\item Chemia II
\item Matematyka II
\item Fizyka II
\item Mechanika
\item CAD
\item Surowce energetyczne polski i świata
\item Wytrzymałość materiałów
\item Fizyka III
\item Technologie informacyjne
\item Elektrotechnika I
\item Metrologia
\item Elektrotechnika II
\item Elektronika I
\item Termodynamika
\item Maszyny elektryczne
\item Pakiet programu MATLAB
\item Technika cieplna
\item Automatyka
\item Odnawialne źródła energii
\item Maszyny energetyczne
\item Przesyłanie energii elektrycznej i techniki zabezpieczeń
\item Podstawy energetyki jądrowej
\item Ochrona środowiska w energetyce
\item Psychologia pracy
\item Eksploatacja maszyn i urządzeń
\item Materiały konstrukcyjne i eksploatacyjne
\item Mikro i makroekonomia
\item Prowadzenie działalności przedsiębiorstwa energetycznego
\item Gospodarka energetyczna
\end{enumerate}
\end{multicols}
\end{document}
