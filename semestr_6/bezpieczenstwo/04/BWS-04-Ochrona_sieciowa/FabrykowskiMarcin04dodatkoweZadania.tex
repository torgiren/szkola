\documentclass[a4paper,11pt]{article}
\usepackage[utf8]{inputenc}
\usepackage{polski}
\usepackage{verbatim}
\author{Marcin Fabrykowski}
\title{Bezpieczeństwo w sieci. Lab 04. Dodatkowe zadania}
\begin{document}
\maketitle
\newpage
\section{Blokowanie routerów w sieci}
Zwykle nie chcemy, aby w naszej były podłączane routery ani tworzone podsieci. Aby to zrobić skorzystamy z parametru TTL w pakietach, aby stacje klienckie były ostatnimi dla pakietów wchodzących do sieci.
\small
\begin{verbatim}
/sbin/iptables -t mangle -A FORWARD -d 192.168.0.0/20 -j TTL --ttl-set 1
\end{verbatim}
\normalsize
zakładamy tutaj ze naszą siecią wewnętrzną jest 192.168.0.0/20.
\section{Po drugiej stronie barykady...}
Lecz czasami znajdujemy sie po drugiej strony barykady, i jesteśmy użytkownikiem który chciałby sobie zrobić podsieć, aby ukryć za natem osoby które nie powinny korzystać z sieci.
\small
\begin{verbatim}
/sbin/iptables -t mangle -A FORWARD -d 172.10.0.0/20 -j TTL --ttl-inc 1
\end{verbatim}
\normalsize
Tutaj zakładamy, że 172.10.0.0/20 to nasza podsieć
\section{Blokowanie stron}
Inna częstą rzeczą, jaką robi administrator to blokowanie poszczególnych stron, np: megaupload.com.\\
Można wtedy postawić vhosta (we własnym zakresie) z informacją o powodach blokowania danej strony i przekierowywać tam wszystkie zapytania kierowane pierwotnie do blokowanego hosta.
\small
\begin{verbatim}
iptables -t nat -D PREROUTING -p tcp -d megaupload.com -j DNAT --to 192.168.0.1:8081
\end{verbatim}
\normalsize
zakładamy tutaj, że  na na 192.168.0.1:8081 jest postawiony nasz vhost z informacją o blokowaniu
\section{Gościnny internet}
Czasami się zdarza, że udostępniamy internet niezidentyfikowanym osobą łącze. Nie chcemy wtedy, aby mieli pełnowartościowy dostęp, a jedynie do www
\small
\begin{verbatim}
iptables -A FORWARD -s 192.168.0.0/24 -p tcp ! --dport 80 -j DROP
\end{verbatim}
\normalsize
Zakładamy tutaj że 192.168.0.0/24 to nasza localna sieć publiczna.
\end{document}

