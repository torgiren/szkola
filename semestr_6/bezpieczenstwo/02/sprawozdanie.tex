\documentclass[a4paper,12pt]{article}
\usepackage{polski}
\usepackage[utf8]{inputenc}
\usepackage{verbatim}
\author{Marcin Fabrykowski}
\title{Bezpieczeństwo w sieci. Lab 02}
\date{}
\begin{document}
\maketitle
\newpage
\section{Odczytywanie niebezpiecznych danych użytkownika}
W tym ćwiczeniu uruchamiamy sniffer i wchodzimy na stronę z formularzem logowania o której wiemy że nie szyfruje danych.\\
Efektem podłuchu jest możliwość odczytania takigo pakietu:
\footnotesize
\verbatiminput{haxite.org}
\normalsize
Jak można zauważyć, w powyższym pakiecie przysyłane sa jawnie dane logowania: "moj\_login" oraz "tajne"
\section{Wireshark}
Chciałbym wspomnieć na wstępie, iż uważam że wstawianie screenshotów z wiresharka mija się z celem.\\
\subsection{Analiza ramek}
We wstępie teoretycznym jest błąd, gdyż protokół ARP rozsyła request po macu FF:FF:FF:FF:FF, gdyż jeszcze nie zna maca odbiorcy. Pozostałe zaobserwowane rzeczy zgadzają się z wstępem teoretycznym.
\subsection{Analiza liczby pakietow}
Łącząc się ze swoją prostą strona, zaobserowałem wymianę kilku pakietów:
\begin{enumerate}
\item 3-way-handshake
\item 2 pakiety requesta i 2 ACK
\end{enumerate}
natomiast, łącząc się z stroną onetu zaobserwowałem
\begin{enumerate}
\item 3-way-handshake
\item ok 20 pakietów segmentowych dla jednej dużej treści strony (+20 tyle samo ACK)
\end{enumerate}
Chciałbym dodać, że wykorzystywaną przeglądarką był links. Przypuszczam również, że oczekiwanym wynikiem była porcja danych którą otrzymałem, a następnie kolejne połączenia dla takich elementów jak obrazki, skrypty JS, oraz CSS.
\subsection{Traceroute}
Wykonanie polecenia traceroute nie dało żadnych przydatnych danych, prawdopodobnie z powodu filtrów na bramie w sieci.\\
Jednak zauważamy w snifferze, że traceroute wysyła dane do badanego hosta, z rosnącymi ttlami. Począwszy od 1. Oczekuje on odpowiedzi od poszczególnych hostów po drodze ze "pakiet umarł" i na tej podstawie próbować będzie badać hosty pośredniczące w komunikacji.
\section{3-way-handshake}
Połączenie inicjowane jest przez klienta. Następuje wymiana pakietów SYN$\rightarrow$SYN,ACK$\rightarrow$ACK, po czym połączenie jest nawiązane.\\
Następnie odbywa się wymiana danych.\\
Zakonczenie połączenia inicjuje klient. Wymiana pakietów: FIN,ACK$\rightarrow$ACK. Następnie od servera: FIN,ACK$\rightarrow$ACK
\section{POP3}
Zauważyłem, że dane do logowania były przesyłane w formie jawnej, co niesie za sobą możliwość bezproblemowego dostępu do skrzynki pocztowej osobom postronnym.
\section{FTP}
Tutaj podobnie jak w POP3, dane do logowania, oraz wykonywane operacje były przesyłane w sposób jawny.
Niestety nie udało mi się pobrać żadnego pliku, dlatego całe moje połączenie odbyło się na porcie 21 i nie zostałem przekierowany na ftp-data (port 20), co zgodnie z moją wiedzą, powinno stać się w przybadku przesyłu danych.
\end{document}
