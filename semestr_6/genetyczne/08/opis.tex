\documentclass[a4,12pt]{article}
\usepackage[utf8]{inputenc}
\usepackage{polski}
\usepackage{graphicx}
\usepackage{float}
\title{Algorytmy genetyczne. Lab 08}
\author{Marcin Fabrykowski}
\date{}
\begin{document}
\maketitle
\newpage
\section{Opis programu}
Celem programu jest znalezienie minimum nieznanej funkcji wykorzystując metodę algorytmów genetycznych.\\
Wyniki wyprowadzane są w funkcji pokoleń, dlatego zastępujemy funkcję \textit{evolve} na pętlę wykorzystującą funkcję \textit{step}
\section{Omówienie badań nad algorytmem}
Badamy zachowanie algorytmu genetycznego przy zmiennych parametrach. Lista badanych parametrów i ich wartości:
\begin{itemize}
\item Prawdopodobieństwo mutacji: 0.01; 0.05; 0.1; 0.2; 0.4;
\item Prawdopodobieństwo krzyżowania: 0.4; 0.5; 0.6; 0.7; 0.8; 0.9;
\item Wielkość populacji: 10; 30; 50; 70; 90; 100; 200;
\item Metody selekcji: roulette; tournament; rank;
\item Metody krzyżowania: onepoint; twopoint; evenodd; unioform;
\end{itemize}
\section{Przedstawienie wyników}
\subsection{Metoda rankingowa}
Zauważamy, że wykorzystanie metody \textit{rankingowej} pozwala otrzymać całkiem dobrą zbieżność już po 15 pokoleniach przy zastosowaniu dużego współczynnika mutacji i małego współczynnika krzyżowania (rys.~\ref{fig:rank_dobre_onepoint}). 
Zauważamy również, że w tym przypadku nie ma dużego wpływu metoda krzyżowania (rys~\ref{fig:rank_dobre_twopoint},\ref{fig:rank_dobre_uniform}). Rozmiar populacji natomiast ma wpływ na polepszenie wyników, przy dobrze dobranych poprzednich parametrach (rys. \ref{fig:rank_dobre_pop50},\ref{fig:rank_dobre_pop100},\ref{fig:rank_dobre_pop200}).
Przy zastosowaniu innego zestawu parametrów, uzyskane wyniki okazują się bardzo niesatysfakcjonujące (rys. \ref{fig:rank_zle_mut},\ref{fig:rank_zle_cross_mut}).
\begin{figure}[b]
\hspace{-3.5cm}
\includegraphics[scale=0.30]{{data_0.4-0.3-100-rank-onepoint}.png}
\caption{Metoda rankingowa}
\label{fig:rank_dobre_onepoint}
\end{figure}
\begin{figure}[b]
\hspace{-3.5cm}
\includegraphics[scale=0.30]{{data_0.4-0.3-100-rank-twopoint}.png}
\caption{Metoda rankingowa - różne krzyżowania}
\label{fig:rank_dobre_twopoint}
\end{figure}
\begin{figure}[b]
\hspace{-3.5cm}
\includegraphics[scale=0.30]{{data_0.4-0.3-100-rank-uniform}.png}
\caption{Metoda rankingowa - różne krzyżowania c.d.}
\label{fig:rank_dobre_uniform}
\end{figure}
\begin{figure}[b]
\hspace{-3.5cm}
\includegraphics[scale=0.30]{{data_0.4-0.3-050-rank-onepoint}.png}
\caption{Metoda rankingowa - różne populacje}
\label{fig:rank_dobre_pop50}
\end{figure}
\begin{figure}[b]
\hspace{-3.5cm}
\includegraphics[scale=0.30]{{data_0.4-0.3-100-rank-onepoint}.png}
\caption{Metoda rankingowa - różne populacje c.d.}
\label{fig:rank_dobre_pop100}
\end{figure}
\begin{figure}[b]
\hspace{-3.5cm}
\includegraphics[scale=0.30]{{data_0.4-0.3-200-rank-onepoint}.png}
\caption{Metoda rankingowa - różne populacje c.d.}
\label{fig:rank_dobre_pop200}
\end{figure}
\begin{figure}[b]
\hspace{-3.5cm}
\includegraphics[scale=0.30]{{data_0.05-0.3-100-rank-onepoint}.png}
\caption{Metoda rankingowa - złe parametry}
\label{fig:rank_zle_mut}
\end{figure}
\begin{figure}[b]
\hspace{-3.5cm}
\includegraphics[scale=0.30]{{data_0.05-0.8-100-rank-onepoint}.png}
\caption{Metoda rankingowa - złe parametry}
\label{fig:rank_zle_cross_mut}
\end{figure}
\subsection{Metoda turniejowa}
Używając metody turniejowej, obserwujemy dwie rodziny parametrów godnych uwagi.
\begin{itemize}
\item Małe krzyżowanie i mała mutacja. Otrzymujemy całkiem dobre wartości najlepszego osobnika, oraz bardzo dobre osobniki medialne. Dobrze zbieżna jest również \textit{zbieżność online} (rys. \ref{fig:tourn_male_cross_onepoint}). Obserwujemy tutaj również nieco lepsze wyniki przy wykorzystaniu krzyżowania \textit{dwupunktowego} (rys.~\ref{fig:tourn_male_cross_twopoint}) niż innych krzyżowań (rys.~\ref{fig:tourn_male_cross_uniform},\ref{fig:tourn_male_cross_evenodd}).
\begin{figure}[b]
\hspace{-3.5cm}
\includegraphics[scale=0.30]{{data_0.01-0.4-100-tournament-onepoint}.png}
\caption{Metoda turniejowa}
\label{fig:tourn_male_cross_onepoint}
\end{figure}
\begin{figure}[b]
\hspace{-3.5cm}
\includegraphics[scale=0.30]{{data_0.01-0.4-100-tournament-twopoint}.png}
\caption{Metoda turniejowa - krzyżowanie dwupunktowe}
\label{fig:tourn_male_cross_twopoint}
\end{figure}
\begin{figure}[b]
\hspace{-3.5cm}
\includegraphics[scale=0.30]{{data_0.01-0.4-100-tournament-uniform}.png}
\caption{Metoda turniejowa - inne krzyżowania}
\label{fig:tourn_male_cross_uniform}
\end{figure}
\begin{figure}[b]
\hspace{-3.5cm}
\includegraphics[scale=0.30]{{data_0.01-0.4-100-tournament-evenodd}.png}
\caption{Metoda turniejowa - inne krzyżowania}
\label{fig:tourn_male_cross_evenodd}
\end{figure}
\item Duża mutacja i duże krzyżowanie. Otrzymujemy wtedy lepszą zbieżność \textit{offline} (rys.~\ref{fig:tourn_duze_mut_onepoint}). Zmiana metody krzyżowania nie niesie za sobą znaczącej poprawy działania algorytmu.
Natomiast zwiększenie mutacji do poziomu 0.04, powoduje pogorszenie otrzymywanych wyników. (rys.~\ref{fig:tourn_za_duze_mut}).
\begin{figure}[b]
\hspace{-3.5cm}
\includegraphics[scale=0.30]{{data_0.2-0.9-100-tournament-onepoint}.png}
\caption{Metoda turniejowa - duża mutacja}
\label{fig:tourn_duze_mut_onepoint}
\end{figure}
\begin{figure}[b]
\hspace{-3.5cm}
\includegraphics[scale=0.30]{{data_0.4-0.9-100-tournament-onepoint}.png}
\caption{Metoda turniejowa - za duża mutacja}
\label{fig:tourn_za_duze_mut}
\end{figure}
\end{itemize}
\subsection{Metoda ruletki}
Metoda ruletki przy małej populacji - 50 osobników, daje lepszą zbieżność online niż metoda rankingowa (rys.~\ref{fig:roulette_duzy_cross}) przy czym jest wydajniejsza, gdyż z samej idei tej metody, nie sortujemy osobników co jest operacją czasochłonną. Ponadto, zbieżność wyników jest lepsza niż w przypadku metody rankingowej (rys.~\ref{fig:rank_dobre_pop50})
\begin{figure}[b]
\hspace{-3.5cm}
\includegraphics[scale=0.30]{{data_0.1-0.9-100-roulette-onepoint}.png}
\caption{Metoda ruletki}
\label{fig:roulette_duzy_cross}
\end{figure}
\section{Wnioski}
\paragraph {Wybór algorytmu.}
Analizując nieznana funkcję różnymi metodami, można stwierdzić że najlepsze wyniki online daje metoda turniejowa z małym prawdopodobieństwem krzyżowania i mutacji, natomiast najlepsze wyniki offline daje metoda rankingowa przy małym prawdopodobieństwie krzyżowania i dużym mutacji.
Wybrałbym jednak metodę turniejową, gdyż ma bardzo dobrą zbieżność online i dobrą offline co czyni ją najbardziej uniwersalną metodą.
\paragraph{Charakter funkcji} Badania wykazały, że wszystkie metody skupiały się w jednym obszarze, co można interpretować, że funkcja posiada jedno minimum w okolicach $x\in(9.035,9.04), y\in(8.665,8.67)$.
\end{document}
