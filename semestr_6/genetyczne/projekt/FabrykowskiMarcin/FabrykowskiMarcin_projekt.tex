\documentclass[12pt,a4paper]{article}
\usepackage[utf8]{inputenc}
\usepackage{polski}
\usepackage{verbatim}
\usepackage{color}
\title{Algorytmy Genetyczne - projekt}
\author{Marcin Fabrykowski}
\begin{document}
\maketitle
\newpage
\tableofcontents
\newpage
\section{Opis problemu}
Mamy dany zbiór N klocków o zadanych wymiarach W i H. Naszym zadaniem jest ułożenie wieży z posiadanych klocków. Zależy nam, aby wieża była jak najwyższa oraz jak najszersza.

Klocki możemy układać tylko z w płaszczyźnie XY, oraz możemy je obracać wokół osi OZ.
\section{Opis rozwiązania problemu dwukryterialności}
W moim projekcie postanowiłem położyć większy nacisk na wysokość wieży, aniżeli na jej szerokość.
Za każdy postawiony klocek, osobnik dostaje 1000pkt, natomiast punkty za szerokość są przyznawane w wartości równej tej szerokości.

Powoduje to, że punktacja za wysokość jest decydująca, a następnie z najwyższych rozwiązań wybieramy najszersze.\\
Punkty za szerokość przyznawane są tylko dla pierwszych 10 klocków.
\section{Szczegółowy opis kodowania}
Osobnik kodowany jest jako tablica numerów klolejno kładzionych klocków.\\
Klocek definiowany jest za pomocą następujących parametrów:
\begin{enumerate}
\item Szerokosc: float itsW
\item Wysokosc: float itsH
\item Obrot: int itsObrot
\item Przesunięcie: float itsX
\end{enumerate}
Przesunięcie klocka kodowane jest w zakresie $<0,1>$, co jest rozumiane jako procent długości klocka spodniego nad którym znajduje się środek ciężkości klocka nadrzędnego.\\
Dla przykładu: dla wartości $0$, środek ciężkości klocka kładzionego, znajduje się nad lewą krawędzią klocka dolnego. Dla wartości $0.5$, klocek górny ma środek ciężkości nad środkiem ciężkości klocka dolnego

Przeliczanie wartości kodowanej przesunięcia na wymiar przesunięcia względnego odbywa się przy użyciu wzoru:
$$x=w_1*dx-\frac{w_2}{2}$$
gdzie: $x$ - pozycja klocka górnego zadanym przez temat formacie\\
$w_1$ - szerokość klocka dolnego\\
$w_2$ - szerokość klocka górnego\\
$dx$ - wartość procentowego przesunięcia odczytywana z osobnika
\section{Szczegółowy opis inicjacji populacji bazowej}
Tworzenie populacji bazowej przebiega w następujący sposób:
\begin{enumerate}
\item Tworzymy osobnika układającego wszystkie klocki w kolejności zadanej z danych wejściowych
\item Zamieniamy miejscami w kolejności dwa losowe klocki. Zamian takich wykonujemy w ilości odpowiadającej rozmiarowi osobnika.
\item Dla każdego klocka losujemy czy będzie on obrócony oraz procent przesunięcia.
\end{enumerate}
\section{Opis użytej funkcji dostosowania}
W funkcji dostosowania sprawdzamy od klocka $i=1$ do $i=popsize$ czy wieża jest stabilna.\\
Zakładamy, że wartość $0$ na osi $OX$ znajduje się na wysokości lewego boku kładzionego klocka. Wtedy zauważamy, że środek jego ciężkości znajduje się w $\frac{w_2}{2}$.\\
Przyjmujemy oznaczenia:\\
$w_1$ - szerokość klocka dolnego\\
$w_2$ - szerokość klocka górnego\\
$dx$ - przesunięcie klocka górnego odczytane z osobnika\\
$k$ - pozycja klocka dolnego względem punktu $0$\\
$l$ - pozycja klocka górnego względem punktu $0$\\
następnie dokładamy klocek od dołu i sprawdzamy czy taki układ jest stabilny, tj. środek ciężkości wyższego znajduje się nad podstawą dolnego.
Jeżeli tak, to obliczamy środek ciężkości układu dwóch klocków i dokładamy kolejny klocek od dołu.
Jeśli środek ciężkości znajduje się nad podstawą dołożonego klocka, uznajemy to za stabilne.
Postępujemy tak, aż dojdziemy do podstawy. Uznajemy wtedy, że wieża do badanego klocka jest stabilna.\\
Powtarzamy wtedy postępowanie biorąc jako pierwszy i+1-wszy klocek.\\
W przypadku nie powodzenia, przerywamy procedurę i przechodzimy do obliczania szerokości.

Szerokość obliczana jest według algorytmu:
\begin{enumerate}
\item przyjmujemy wartość min jako lewy bok podstawy
\item przyjmujemy wartość max jako prawy bok podstawy
\item przechodzimy do następnego klocka
\item sprawdzamy czy pozycja lewej krawędzi jest mniejsza niż min. Jeśli tak to min=pozycja lewego boku
\item jeśli pozycja prawego boku jest większa niż max, to max=pozycja prawego boku
\item powtarzamy dla 10 klocków licząc od podstawy
\end{enumerate}
do wartości dostosowania dodajemy szerokość wierzy wyrażoną jako $max-min$.
\section{Opis użytych metod krzyżowania, mutacji oraz selekcji}
\subsection{Metody krzyżowania}
Zakładamy przyjęcie następujących stałych:
\begin{enumerate}
\item wielkość populacji: 500
\item liczba generacji: 500
\item metoda selekcji: RankSelector
\item prawdopodobieństwo mutacji: 0.01
\item prawdopodobieństwo krzyżowania: 0.8
\end{enumerate}
Badanie metod krzyżowania przedstawia poniższa tabela\\
\begin{tabular}{c|c|c|c}
&OrderCrossover&CycleCrossover&PartialMatchCrossover\\
\hline
czas&\textcolor{red}{20s}&--&21s\\
\hline
klockow&230&--&\textcolor{red}{238}\\
\hline
szerokosc&29&--&\textcolor{red}{35}\\
\end{tabular}\\
Powyższe zestawienie skłania mnie do wybrania metody PartialMatchCrossover, ponieważ metoda daje lepsze wyniki przy nieznacznie dłuższym czasie wykonywania.
\subsection{Metody mutacji}
Wykorzystałem własną funkcję mutacji.\\
Testowane były 4 różne metody mutacji:
\begin{enumerate}
\item zamieniająca losowe dwa klocki
\item zamieniająca losowe dwa klocki oraz zmiana obrotu klocka
\item zamieniająca losowe dwa klocki oraz zmiana obrotu klocka oraz zmiana przesunięcia klocka
\item zamieniająca losowe dwa klocki oraz zmiana przesunięcia klocka
\end{enumerate}
Testy wykazały, że wykorzystanie funkcji zmieniających obrót i/lub przesunięcie powodowało drastyczny spadek ilości ułożonych klocków do wartości rzędu 5-30 klocków, dlatego wybrana została funkcja zmieniająca jedynie kolejność klocków.
\subsection{Metody selekcji}
Zakładamy przyjęcie następujących stałych:
\begin{enumerate}
\item wielkość populacji: 500
\item liczba generacji: 500
\item metoda krzyżowania: PartialMatchCrossover
\item prawdopodobieństwo mutacji: 0.01
\item prawdopodobieństwo krzyżowania: 0.8
\end{enumerate}
Testy wykazały następujące wyniki\\
\begin{tabular}{c|c|c|c}
&RankSelector&TournamentSelector&RouletteWheelSelector\\
\hline
czas&23s&\textcolor{red}{6s}&\textcolor{red}{6s}\\
\hline
klockow&\textcolor{red}{252}&7&36\\
\hline
szerokosc&\textcolor{red}{34}&31&24\\
\end{tabular}\\
Zauważamy, że metoda rankingowa daje znacząco lepsze efekty, dlatego zdecydowałem się na tę właśnie metodę.
\section{Podsumowanie wybranych parametrół}
Ostatecznie zostały wybrane następujące parametry algrorytmu:
\begin{itemize}
\item Metoda krzyżowania: PartialMatchCrossover\\
Uzasadnienie w odpowiedniej sekcji
\item Metoda selekcji: RankSelector\\
Uzasadnienie w odpowiedniej sekcji
\item Metoda mutacji: własna metoda\\
Uzasadnienie w odpowiedniej sekcji
\item Wielkość populacji: 1000\\
Testy wykazały poprawę wyników przy zwiększeniu populacji do 1000 osobników
\item Ilość pokoleń: 1000\\
Testy wykazały poprawę wyników przy zwiększeniu liczby pokoleń do 1000
\item Prawdopodobieństwo mutacji: 0.01\\
Większe wartości powodują znaczne pogorszenie wyników
\item Prawdopodobieństwo krzyżowania 0.8\\
Wartości oscylujące około 0,8 - 0.9 nie powodują znacznych zmian w wynikach
\end{itemize}
\end{document}
