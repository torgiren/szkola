\documentclass[12pt,a4paper]{article}
\usepackage[utf8]{inputenc}
\usepackage{polski}
\usepackage{verbatim}
\title{Algorytmy Genetyczne - projekt}
\author{Marcin Fabrykowski}
\begin{document}
\maketitle
\newpage
\tableofcontents
\newpage
\section{Opis problemu}
Mamy dany zbiór N klocków o zadanych wymiarach W i H. Naszym zadaniem jest ułożenie wieży z posiadanych klocków. Zależy nam, aby wieża była jak najwyższa oraz jak najszersza.

Klocki możemy układać tylko z w płaszczyźnie XY, oraz możemy je obracać wokół osi OZ.
\section{Opis rozwiązania problemu dwukryterialności}
W moim projekcie postanowiłem położyć większy nacisk na wysokość wieży, aniżeli na jej szerokość.
Za każdy postawiony klocek, osobnik dostaje 1000pkt, natomiast punkty za szerokość są przyznawane w wartości równej tej szerokości.

Powoduje to, że punktacja za wysokość jest decydująca, a następnie z najwyższych rozwiązań wybieramy najszersze.
\section{Szczegółowy opis kodowania}
Osobnik kodowany jest jako tablica numerów klolejno kładzionych klocków.\\
Klocek definiowany jest za pomocą następujących parametrów:
\begin{enumerate}
\item Szerokosc: float itsW
\item Wysokosc: float itsH
\item Obrot: int itsObrot
\item Przesunięcie: float itsX
\end{enumerate}
Przesunięcie klocka kodowane jest w zakresie $<0,1>$, co jest rozumiane jako procent długości klocka spodniego nad którym znajduje się środek ciężkości klocka nadrzędnego.\\
Dla przykładu: dla wartości $0$, środek ciężkości klocka kładzionego, znajduje się nad lewą krawędzią klocka dolnego. Dla wartości $0.5$, klocek górny ma środek ciężkości nad środkiem ciężkości klocka dolnego

Przeliczanie wartości kodowanej przesunięcia na wymiar przesunięcia względnego odbywa się przy użyciu wzoru:
$$x=w_1*dx-\frac{w_2}{2}$$
gdzie: $x$ - pozycja klocka górnego zadanym przez temat formacie\\
$w_1$ - szerokość klocka dolnego\\
$w_2$ - szerokość klocka górnego\\
$dx$ - wartość procentowego przesunięcia odczytywana z osobnika
\section{Szczegółowy opis inicjacji populacji bazowej}
Tworzenie populacji bazowej przebiega w następujący sposób:
\begin{enumerate}
\item Tworzymy osobnika układającego wszystkie klocki w kolejności zadanej z danych wejściowych
\item Zamieniamy miejscami w kolejności dwa losowe klocki. Zamian takich wykonujemy w ilości odpowiadającej liczbie osobników w populacji
\item Dla każdego klocka losujemy czy będzie on obrócony oraz procent przesunięcia.
\end{enumerate}
\section{Opis użytej funkcji dostosowania}
W funkcji dostosowania sprawdzamy od klocka $i=1$ do $i=popsize$ czy wieża jest stabilna.\\
Zakładamy, że wartość $0$ na osi $OX$ znajduje się na wysokości lewego boku kładzionego klocka. Wtedy zauważamy, że środek jego ciężkości znajduje się w $\frac{w_2}{2}$.\\
Przyjmujemy oznaczenia:\\
$w_1$ - szerokość klocka dolnego\\
$w_2$ - szerokość klocka górnego\\
$dx$ - przesunięcie klocka górnego odczytane z osobnika\\
$k$ - pozycja klocka dolnego względem punktu $0$\\
$l$ - pozycja klocka górnego względem punktu $0$\\
\end{document}
