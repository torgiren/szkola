\documentclass[a4paper,12pt]{article}
\usepackage{polski}
\usepackage[utf8]{inputenc}
\author{Marcin Fabrykowski}
\title{Algorytmy genetyczne. Lab 10}
\date{}
\begin{document}
\maketitle
\newpage
\section{Opis problemu}
Naszym zadaniem jest dobranie takiej kolejności prętów, aby tworząc z każdych trzech kolejnych prętów utworzyć trójkąt.\\
Dodatkowo zależy nam, aby liczba utworzonych trójkątów była jak największa. Ponadto zależy nam aby odchylenie pól powstałych trójkątów było jak najmniejsze.
\section{Kodowanie}
Listę prętów wczytujemy z pliku podanego jako argument wywołania programu. Wczytujemy ją do tablicy.\\
Kodujemy rozwiązanie jako tablicę indeksów tablicy prętów.
\section{Ocenianie rozwiązania}
Użyty przeze mnie algorytm przyznawania punktów w funkcji przystosowania przedstawia się następująco:
\begin{enumerate}
\item Dla stu trójek prętów, sprawdzamy czy można zbudować trójkąt
\item Jeśli można zbudować, dodajemy do wyniku dla badanego osobnika 100pkt, oraz obliczamy pole tego trójkąta i dodajemy je do listy pól.
\item Obliczamy średnie pole powstałych trójkątów oraz odchylenie standardowe.
\item Od wyniku osobnika, odejmujemy wartość odchylenia standardowego pomnożona przez empirycznie wyznaczony współczynnik równy $2.5$.
\end{enumerate}
Szukamy osobnika o najwyższym współczynniku dostosowania.
\section{Metody selekcji i krzyżowania}
Testowano różne metody selekcji i krzyżowania. Poniższa tabela przedstawia średnie wyniki otrzymane dla 10 uruchomień programu.\\
\begin{center}
\begin{tabular}{c|c|c|c|c}
&&Rank&Tournament&Roulette\\ \hline
PartialMach&trójkątów&97,4&66,2&61,4\\
&odchylenie&53,6&130,3&135,5\\ \hline
Cycle&trójkątów&98,0&X&X\\
&ochylenie&54,9&X&X\\ \hline
Order&trójkątów&97,8&62,2&58,9\\
&ochylenie&58,5&132,8&137,4\\
\end{tabular}\\
\small
X - nieotrzymanie rozwiązania w~rozsądnym czasie
\normalsize
\end{center}
Widzimy, że metoda rankingowa daje najlepsze wyniki, bez większego wpływu metody krzyżowania.\\
Decydujemy się na wybór metody krzyżowania \textit{Cycle Crossover}, ponieważ daje najlepsze wyniki, a~przy tym działa czterokrotnie szybciej niż dwie pozostałe metody krzyżowania.
\end{document}
