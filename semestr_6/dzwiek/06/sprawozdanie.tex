\documentclass[10pt,a4paper]{article}
\usepackage{polski}
\usepackage[utf8]{inputenc}
\title{Dźwięk i muzyka w systemach komputerowych - laboratorium 06}
\author{Marcin Fabrykowski}
\date{}
\begin{document}
\maketitle
\newpage
\begin{enumerate}
\item Sygnał prostokątny\\
Naszym zadaniem jest wygenerowanie sygnału prostokątnego, a następnie skompresowanie go z bitratem 128, 64 oraz 32.\\
Wyniki widać na załączonych rysunkach.
\item Sygnał piłokształtny\\
Naszym zadaniem jest wygenerowanie sygnału piłokszatłtnego, a następnie skompresowanie go z bitratem 128, 64 oraz 32.\\
Wyniki widać na załączonych rysunkach.
\item Sygnał rzeczywisty\\
Naszym zadaniem jest wczytanie dowolnego pliku dźwiękowego, a następnie skompresowanie go z bitratem 128, 64 oraz 32.\\
Wyniki widać na załączonych rysunkach.
\end{enumerate}
\end{document}
