\documentclass[a4paper,12pt]{article}
\usepackage[utf8]{inputenc}
\usepackage{polski}
\usepackage{listings}
\author{Marcin Fabrykowski}
\title{Modelowanie procesów fizycznych\\Lab 07}
\begin{document}
\maketitle
\newpage
\section{Stężenie SO2}
Wartość zależności stężenia od stanu atmosfery:\\
Dla stanu: a= 0.888 b= 1.284  m= 0.08 u= 3\\
4.75122451477\\
Dla stanu: a= 0.865 b= 1.108  m= 0.143 u= 5\\
9.80991218835\\
Dla stanu: a= 0.845 b= 0.978  m= 0.196 u= 8\\
17.2593528244\\
Dla stanu: a= 0.818 b= 0.822  m= 0.27 u= 11\\
45.8658110373\\
Dla stanu: a= 0.784 b= 0.66  m= 0.363 u= 5\\
373.950857633\\
Dla stanu: a= 0.756 b= 0.551  m= 0.44 u= 4\\
795.018638357\\
\section{Kod programu}
\footnotesize
\lstinputlisting[language=python]{lab07.py}
\normalsize
\section{Wnioski}
Zauważamy, że stężenie dwutlenku siarki bardzo mocno zależy od aktualnego stanu atmosfery.
\end{document}
