\documentclass[a4paper, 12pt]{article}
\usepackage[utf8]{inputenc}
\usepackage{polski}
\usepackage{graphicx}
\author{Marcin Fabrykowski}
\title{Analiza i przetwarzanie obrazu\\Lab 07}
\begin{document}
\maketitle
\newpage
\section{Wstęp}
Celem ćwiczenia było opracowanie metod wykrywania liter na obrazach
\section{Wykonanie ćwiczenia}
Metoda pierwsza charakteryzuj się szybkością, jednak jest wrażliwa na litery znajdujące się na tej samej linii pionowej, np: lj, natomiast metoda 2 potrafi takie znaki oddzielić, jednak działa zauważalnie wolniej.
\section{Wnioski}
Przy rozpoznawaniu liter większe znaczenie ma dokładność aniżeli szybkość (bez szczególnych przypadków), dlatego metoda nr~2 ma moim zdaniem lepsze zastosowanie niż metoda 1.
\end{document}
