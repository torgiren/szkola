\documentclass[12pt, a4paper]{article}
\usepackage{polski}
\usepackage[utf8]{inputenc}
\usepackage{graphicx}
\usepackage{listings}
\author{Marcin TORGiren Fabrykowski}
\title{AIPO - Laboratorium 2}
\begin{document}
\lstset{language=Python, tabsize=2, inputencoding=utf8, basicstyle=\footnotesize}
\maketitle
\newpage
\section{Teoria}
\subsection{Filtr rozmywający}
Filtr splotowy uśredniający (dolnoprzepustowy) - działanie tego filtru polega na ustawieniu wartości danego piksela na podstawie wartości pikseli znajdujących się w najbliższym otoczeniu tego piksela. Wynikiem jest uzyskanie rozmycia obrazu wejściowego.
\subsection{Filtr wyostrzający}
Filtr splotowy wyostrzający (górnoprzepustowy) - służy do wzmocnienia szczegółów o dużej częstotliwości. Wynikiem jest poprawa ostrości oraz kontrastu obrazu, jednak wzmacnia również szumy. Często stosowane po silnej filtracji uśredniającej, aby przywrócić ostrość obrazu.
\subsection{Przetwarzany obraz}
Obraz źródłowy znajduje się na rys.~\ref{fig:src}
\begin{figure}[p]
\includegraphics{tux}
\caption{Obraz źródłowy}
\label{fig:src}
\end{figure}
\newpage
\section{Analiza obrazu}
\subsection{Skalowanie}
Na rys \ref{fig:down} przedstawiono skalowanie w dół w skali 0.4, natomiast na rys \ref{fig:up} skalowanie w górę w skali 2.5.

\begin{figure}[p]
\includegraphics{scale_down}
\caption{Skalowanie w dół - 0.4}
\label{fig:down}
\end{figure}

\begin{figure}[p]
\includegraphics{scale_up}
\caption{Skalowanie w górę - 2.5}
\label{fig:up}
\end{figure}
\subsection{Progowanie}
Na rysunku \ref{fig:global} przedstawiono progowanie z użytym globalną wartością progowania.
\begin{figure}[p]
\includegraphics{global}
\caption{Progowanie globalne}
\label{fig:global}
\end{figure}
Na rysunkach \ref{fig:local_5}, \ref{fig:local_11}, \ref{fig:local_15}, \ref{fig:local_21}, \ref{fig:local_25} przedstawiono progowanie localne z otoczeniami odpowiednio: 5, 11, 15, 21, 25.\\
\begin{figure}[p]
\includegraphics{local_5}
\caption{Progowanie lokalne, otoczenie 5}
\label{fig:local_5}
\end{figure}

\begin{figure}[p]
\includegraphics{local_11}
\caption{Progowanie lokalne, otoczenie 11}
\label{fig:local_11}
\end{figure}
\begin{figure}[p]
\includegraphics{local_15}
\caption{Progowanie lokalne, otoczenie 15}
\label{fig:local_15}
\end{figure}
\begin{figure}[p]
\includegraphics{local_21}
\caption{Progowanie lokalne, otoczenie 21}
\label{fig:local_21}
\end{figure}
\begin{figure}[p]
\includegraphics{local_25}
\caption{Progowanie lokalne, otoczenie 25}
\label{fig:local_25}
\end{figure}
Na rysunkach \ref{fig:mixed_15}, \ref{fig:mixed_25}, \ref{fig:mixed_35} na których przedstawiono progowanie mieszane dla otoczenia 15 i odchyleniu średniej odpowiednio: 15, 25, 35.\\
\begin{figure}[p]
\includegraphics{mixed_15}
\caption{Progowanie mieszane, odchylenie 15}
\label{fig:mixed_15}
\end{figure}
\begin{figure}[p]
\includegraphics{mixed_25}
\caption{Progowanie mieszane, odchylenie 25}
\label{fig:mixed_25}
\end{figure}
\begin{figure}[p]
\includegraphics{mixed_35}
\caption{Progowanie mieszane, odchylenie 35}
\label{fig:mixed_35}
\end{figure}
\subsection{Filtry splotowe}
Na rysunkach \ref{fig:splot_1_1}, \ref{fig:splot_1_2}, \ref{fig:splot_1_3} przedstawiono filtrowanie obrazu następującymi filtrami rozmywającymi:\\
\begin{tabular}{c|c|c}
1&1&1\\ \hline
1&1&1\\ \hline
1&1&1\\
\end{tabular},
\begin{tabular}{c|c|c}
1&1&1\\ \hline
1&2&1\\ \hline
1&1&1\\
\end{tabular},
\begin{tabular}{c|c|c}
1&2&1\\ \hline
2&4&2\\ \hline
1&2&1\\
\end{tabular}\\
\begin{figure}[p]
\includegraphics{splot_1_1}
\caption{Filtrowanie rozmywające 1}
\label{fig:splot_1_1}
\end{figure}
\begin{figure}[p]
\includegraphics{splot_1_2}
\caption{Filtrowanie rozmywające 2}
\label{fig:splot_1_2}
\end{figure}
\begin{figure}[p]
\includegraphics{splot_1_3}
\caption{Filtrowanie rozmywające 3}
\label{fig:splot_1_3}
\end{figure}
Na rysunkach \ref{fig:splot_2_1}, \ref{fig:splot_2_2}, \ref{fig:splot_2_3} przedstawiono filtrowanie obrazu następującymi filtrami wyostrzającymi:\\
\begin{tabular}{c|c|c}
0&-1&0\\ \hline
-1&5&-1\\ \hline
0&-1&0\\
\end{tabular},
\begin{tabular}{c|c|c}
-1&-1&-1\\ \hline
-1&9&-1\\ \hline
-1&-1&-1\\
\end{tabular},
\begin{tabular}{c|c|c}
1&-2&1\\ \hline
-2&5&-2\\ \hline
1&-2&1\\
\end{tabular}\\
\begin{figure}[p]
\includegraphics{splot_2_1}
\caption{Filtrowanie wyostrzające 1}
\label{fig:splot_2_1}
\end{figure}
\begin{figure}[p]
\includegraphics{splot_2_2}
\caption{Filtrowanie wyostrzające 2}
\label{fig:splot_2_2}
\end{figure}
\begin{figure}[p]
\includegraphics{splot_2_3}
\caption{Filtrowanie wyostrzające 3}
\label{fig:splot_2_3}
\end{figure}
\newpage
\section{Kod programu}
\lstinputlisting{image_anal_ascii.py}
\section{Wnioski}
Jakoś skalowanie w górę można poprawić poprzez obliczanie wartości średnich dla nowych pixeli zamiast kopiować jeden pixel wielokrotnie.\\
Najlepszy wynik progowowanie dała metoda mieszana z odchyleniem 35. Usuwa ona wszystkie niepotrzebne detale i pozostawia najważniejsze elementy obrazu, tj. kształt pingwina, a usuwa drobne przebłyski na smokingu i stopach.\\
Wszystkie 3 filtry rozmywające dają podobny efekt wizualny, oraz nie różnią się zbytnio złożonością obliczeniową, dlatego uważam że wszystkie one są równorzędne.\\
Moim zdaniem filtr wyostrzający 2 daje najlepsze wyniki. Można to zauważyć np. przy uwydatnieniu drobnego cienia świetlnego na rękach który mógłby zostać niezauważony na oryginalnym obrazku. Podobnie jest z prawą brwią pingwina - praktycznie niezauważalna na oryginalnym obrazku zostaje wyraźnie przedstawiona przy filtrze nr. 2
\end{document}
