\documentclass[a4paper, 12pt]{article}
\usepackage[utf8]{inputenc}
\usepackage{polski}
\author{Marcin TORGiren Fabrykowski}
\title{Analiza i przetwarzanie obrazu\\Projekt}
\begin{document}
\maketitle
\newpage
\tableofcontents
\newpage
\section{Opis problemu}
Naszym zadaniem jest przeprowadzenie analizy obrazu pod kątem znalezienia i rozpoznania zawartych na nim tekstów.
\section{Wymagania}
Do poprawnego działania programu, potrzebne są następujące biblioteki:
\begin{itemize}
\item PIL
\item scipy.misc
\item shutil
\item numpy
\end{itemize}
\section{Algorytm nauki}
Proces nauki przeprowadzany jest według następującego algorytmu:
\begin{enumerate}
\item Wybranie czcionki
\item Wygenerowanie obrazka zawierającego alfabet
\item Segmentacja obrazu
\item Posortowanie obrazu po współrzędnych X-owych
\item Przypisanie kolejnym elementom kolejne litery alfabetu
\item Przeskalowanie obrazów oraz przeniesienie do katalogu ze znanymi literami
\item Całość powtórzyć dla wszystkich testowych czcionek
\end{enumerate}
\section{Segmentacja}
Proces segmentacji przebiega według następującego algorytmu:
\begin{enumerate}
\item Znalezienie pierwszego czarnego piksela
\item Dodanie do kolejki wszystkich jego czarnych sąsiadów
\item Dodanie aktualnego piksela do listy pikseli tworzących literę oraz usunięcie go z obrazka
\item Powtórzenie pkt 2-3 dla wszystkich pikseli w kolejce
\item Po wyczyszczeniu kolejki, powtórzenie algorytmu od pkt. 1
\end{enumerate}
\section{Rozpoznanie}
W celu rozpoznania litery, zastosowana została metoda "najmniejszej różnicy". Metoda ta polega na reprezentacji badanego obrazka jako macierzy dwuwymiarowej, a następnie przeprowadzeniu operacji odejmowania macierzowego z każdą macierzą reprezentującą litery poznane w procesie nauki.
Dla każdej tak powstałej macierzy wynikowej, obliczane są wartości bezwzględne w ich elementów, a następnie sumowane.
Tak otrzymana suma, jest "miarą niedopasowania" litery.
Wybierana jest litera która ma najmniejszą wartość niedopasowania
\end{document}
